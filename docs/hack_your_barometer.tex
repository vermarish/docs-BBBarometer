\documentclass[12pt]{article}
\usepackage{amsmath, amsfonts, amsthm, amssymb}
\usepackage{fullpage}
\usepackage{setspace}
\usepackage{graphicx}
\usepackage{tikz}
\usepackage{listings}
\usetikzlibrary{decorations,arrows}

\doublespacing

%opening
%\title{}
%\author{Rishabh Verma}

\begin{document}

%\maketitle
%\begin{abstract}
%\end{abstract}

\section{Data available:}
\begin{enumerate}
	\item Accelerometer data: measures all forces (include gravity) in three dimensions. Note that this is auto-processed into orientation data. I don't think this will be useful because hand motions will dominate.
	\item Gyroscope data: measures rate of rotation in all three physical axes. Highly worth looking into.
	\item Barometer: one-dimensional, determines device deformation, may determine radial location of input
\end{enumerate}

\section{Possible models}
\begin{enumerate}
	\item Peaks in pressure sensor will indicate time-series data points of when the device is pressed. 
\end{enumerate}

\section{Procedure}
\begin{enumerate}
	\item Create application where accelerometer/barometer data can be recorded and touch events can be logged
	\begin{enumerate}
		\item start/stop for logging accelerometer/barometer.
			\begin{enumerate}
				\item Make a button with text that changes when pressed.
				\item Figure out how to log data in external files.
				\item Log it.
				\item 
			\end{enumerate}
		\item within the start/stop, record touches
			\begin{enumerate}
				\item Figure out how to record touch information. Do I need some sort of canvas?
			\end{enumerate}
	\end{enumerate}
	\item Log some things, upload to computer.

	\begin{enumerate}
		\item Start by placing a dot at a random location that moves when a tap is detected.
	\end{enumerate}
	\item From the sample datastream, figure out how to isolate peaks. 
	\item Assign a normed score for each peak.
	\begin{enumerate}
		\item Perhaps the normed score could be the height, or the ratio of peak height to width (width determined by half height). Maybe weight this ratio, let it be a parameter.
	\end{enumerate}
	\item Create a heatmap of peak norms and pixels, and determine viability of project. Is there a working polar-coordinate proxy? Or will we have to do more abstract machine learning stuff?
	\item Once a peak detection algorithm has been identified, implement it on the device so that it logs only the peaks instead of all noise (for data reduction)
	\item Create application where digits can be inputted and recorded.
	\item Include accelerometer/barometer write-to-disk.
	\item Figure out peak detection; need to log tuples
	\begin{enumerate}
		\item Accelerometer curve
		\item Barometer curve
		\item 
	\end{enumerate}
	\item Dump it all into SQL on desktop
	\item Do the machine learning
	\begin{enumerate}
		\item Start by creating a heatmap of 
	\end{enumerate}
	\item With the model, attempt to identify digits from accelerometer/barometer
	\begin{enumerate}
		\item Create application with grid
		\item User inputs data for SOME PERIOD OF TIME
		\item 
	\end{enumerate}
\end{enumerate}

\end{document}
